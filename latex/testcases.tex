\chapter{Global Test Cases}
During the development process continuous integration tools will be used to run at least the following test cases: \\

\begin{longtable}{|p{1cm} | p{15cm}|}
  \hline
  /T10/ & Read all global variables.\\
  \hline
  /T20/ & Write to all global variables that are not constant.\\
  \hline
  /T30/ & Write to all global variables that are constant and ensure the correct Exception is thrown.\\
  \hline
  /T40/ & Create instances of all classes with a public constructor.\\
  \hline
  /T50/ & Call all methods of these objects with valid parameters, where valid means that the data type is correct, a method throwing an exception due to invalid input shall be considered as a passed test, a crash due to e.g. invalid memory read shall be considered as a failed test.\\
  \hline
  /T60/ & Read all public member variables of these classes.\\
  \hline
  /T70/ & Write to all public member variables of these classes that are not constant.\\
  \hline
  /T80/ & Write to all public member variables of these classes that are constant and ensure the correct exception is thrown.\\
  \hline
  /T90/ & Create instances of classes with private constructors and ensure the correct exception is thrown.\\
  \hline
  /T100/ &  Apply the test cases described in /T60/ to /T90/ to static members and methods. Ensure the correct exceptions are thrown.\\
  \hline
  /T110/ & Let Cling evaluate strings of valid C++ code during run time and ensure ROOT's state wad modified as expected.\\
  \hline
  /T120/ & Let Cling evaluate strings of syntactically invalid C++ code during run time and ensure the right exceptions are thrown and forwarded correctly.\\
  \hline
  /T130/ & Let Cling evaluate strings of semantically invalid C++ code during run time and ensure the right exceptions are thrown and forwarded correctly.\\
  \hline
\end{longtable}
