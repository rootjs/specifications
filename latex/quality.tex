\chapter{Quality assurance}
We will ensure correct functionality by writing unit tests for every class. Test cases will be run after every git push using continuous integration tools.\\ \\
Furthermore a function test will be executed, testing if all exposed elements are really accessible and working (see global test cases).\\ \\
We will use the \textit{Github issue tracker}\footnote{\url{https://github.com/rootjs/rootjs/issues}} to track all issues we encounter, even issues that the reporter fixes himself should be tracked.
Closing an issue in the issue tracker is only allowed if a test case is provided that both fails before and succeeds after the fix. This ensures that the same bug is not introduced multiple times by different people.\\ \\
We will check the test coverage after every push to the Github repository. If the coverage decreases on a special method its developer needs to check if there is a branch that is not covered yet and possibly add new test cases for it.
