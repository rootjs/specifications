\chapter{Product interface and functions}
The rootJS bindings will not have a user interface, neither a graphical user interface nor a command line interface.
This section will therefore specify the basic API of rootJS.


\begin{longtable}{|p{1cm} | p{15cm}|}
  \hline
  /I10/ & The rootJS module will expose a JavaScript object containing all accessible ROOT variables, functions and classes.\\
  \hline
  /I20/ & Exposed variables may contain scalar values, in which case they will be accessible in their JavaScript counterparts.\\
  \hline
  /I30/ & Exposed variables may be objects, which are recursively converted to JavaScript objects until only scalar values remain.\\
  \hline
  /I40/ & Exposed variables may be enums, in which case the identifier of the currently selected value is returned instead of the corresponding integer.\\
  \hline
  /I50/ & Every exposed method will be accessible via a proxy method, which handles parameter overloading, as JavaScript does not support overloading. If there is no method to handle the passed arguments, an exception will be thrown.\\
  \hline
  /I55/ & A method may be called with an additional callback method that will be called after the original method has been executed.\\
  \hline
  /I60/ & Exposed classes will be accessible via constructors returning the corresponding objects. A constructor will be accessible through a proxy function to support parameter overloading. If there is no method to handle the passed arguments, an exception will be thrown.\\
  \hline
  /I65/ & A constructor can be called with an additional callback method that will be executed after the object has been constructed.\\
  \hline
  /I70/ & The classes are encapsulated in their namespaces from ROOT. Each namespace is an object containing namespaces or class constructors.\\
  \hline
  /I80/ & Exceptions thrown by ROOT will be forwarded to JavaScript and can be treated as normal exceptions by JavaScript.\\
  \hline
  /I90/ & Global variables are accessible via getter and setter methods to ensure their values are kept in sync with the ROOT framework.\\
  \hline
\end{longtable}
