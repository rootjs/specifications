\chapter{Appendix}
\section{Glossary}
\paragraph{Application programming interface (API)}
A collection of tools, libraries and documentation for creating software. APIs expose functionality through specified interfaces, which allows to develop independent from the specific implementation.
\paragraph{Asynchronous I/O}
Using asynchronous I/O, an application does not have to wait for the data transmission to finish, but can continue in execution. This prevents leaving the CPU idle, as I/O operations are usually significantly slower than other tasks.
\paragraph{C++}
A general purpose, object oriented programming language and one of the most widely used programming languages.

\paragraph{Compiler}
A computer program that translates source code into another language.
\paragraph{Cling}
An interactive C++ interpreter with a command line prompt and uses a just-in-time compiler.

\paragraph{enums}
A data structure that contains named elements that can be compared to each other. Each element maps to a constant value.

\paragraph{CERN}
The European Organization for Nuclear Research

\paragraph{Exception}
Exception handling is the process of responding to the occurrence, during computation, of exceptions – anomalous or exceptional conditions requiring special processing - often changing the normal flow of program execution.\footnote{\url{https://en.wikipedia.org/wiki/Exception_handling}}

\paragraph{Framework}
Extendable, generic software to be extended by user-written modules.

\paragraph{Github}
A host for Git repositories.

\paragraph{Git}
A VCS  Version Control System)

\paragraph{Google V8 JavaScript}
An open-source JavaScript interpreter.

\paragraph{HTTP}
The HyperText Transfer Protocol, used for data communication on the internet.

\paragraph{Interpreter}
A computer program that directly executes written instructions, without compiling them first.

\paragraph{JavaScript}
An interpreted programming or script language created by Netscape.

\paragraph{JavaScript Object Notation (JSON)}
A data format which is easy to read and write for humans.

\paragraph{Just-in-time (JIT) compilation}

\paragraph{Language bindings}
Expose an API to another programming language. This enables usage of low-level functionality in higher-level environments (such as the ROOT API in Node.js using the rootJS bindings).

\paragraph{Large Hadron Collider (LHC)}
World's largest particle collider located in CERN

\paragraph{Linux}
A free and open source computer operating system.

\paragraph{Low Level Virtual Machine (LLVM)}
A compiler infrastructure written in C++.

\paragraph{Mac}
A Unix based operating system created by Apple.

\paragraph{Method overloading}
Declaring a method multiple times within the same namespace using different parameters

\paragraph{Method signature}
Defines the data a method takes in and puts out. 

\paragraph{Node.js}
A runtime environment for developing server-side web applications written in Javascript.

\paragraph{Object}

\paragraph{Operating System}
A piece of software managing software and hardware resources, input/output, and also controls the overall operation of the computer system.
\paragraph{Platform}

\paragraph{ROOT}
A framework for data processing, particularly for particle physicists. ROOT was developed by CERN C++.


\paragraph{RPC}
Remote procedure call. Execute a routine in another address space without coding details for the remote interaction

\paragraph{Stream}

\paragraph{Web server}

\paragraph{Windows}
A DOS based operating system created by Microsoft.

\paragraph{VCS}
Version Control Systems are used to track and (if necessary) revert changes made on a software project.

\pagebreak[4]

\section{Links}

The Github repository for this document can be found at \url{https://github.com/rootjs/specifications}. \\
The repository for the project itself is located at \url{https://github.com/rootjs/rootjs}.
