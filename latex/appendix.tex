\chapter{Appendix}
\section{Glossary}
\paragraph{Application programming interface (API)}
A collection of tools, libraries and documentation for creating software. APIs expose functionality through specified interfaces, which allows to develop independent from the specific implementation.
\paragraph{Asynchronous I/O}
Using asynchronous I/O, an application does not have to wait for the data transmission to finish, but can continue in execution. This prevents leaving the CPU idle, as I/O operations are usually significantly slower than other tasks.
\paragraph{C++}
A general purpose, object oriented programming language and one of the most widely used programming languages.
\paragraph{Cling}
An interactive C++ interpreter with a command line prompt and uses a just-in-time compiler.
\paragraph{enums}
A data structure that contains named elements that can be compared to each other. Each element maps to a constant value.
\paragraph{European Organization for Nuclear Research (CERN)}
\paragraph{Exception}
\paragraph{Expose}
\paragraph{Framework}
\paragraph{Github}
\paragraph{Google V8 JavaScript}
\paragraph{HTTP}
\paragraph{Input/Output (I/O)}
\paragraph{Interpreter}
\paragraph{JavaScript}
An interpreted programming or script language created by Netscape.
\paragraph{JavaScript Object Notation (JSON)}
A data format which is easy to read and write for humans.
\paragraph{Just-in-time (JIT) compilation}
\paragraph{Language bindings}
Expose an API to another programming language. This enables usage of low-level functionality in higher-level environments (such as the ROOT API in Node.js using the rootJS bindings).
\paragraph{Large Hadron Collider (LHC)}
world's largest particle collider located in
\paragraph{Linux}
A free and open source computer operating system.
\paragraph{Low Level Virtual Machine (LLVM)}
A compiler infrastructure written in C++.
\paragraph{Mac}
A Unix based operating system created by Apple.
\paragraph{Method overloading}
declaring a method multiple times within the same namespace using different parameters
\paragraph{Method signature}

\paragraph{Node.js}
A runtime environment for developing server-side web applications written in Javascript.

\paragraph{Object}

\paragraph{Operating System}
A piece of software managing software and hardware resources, input/output, and also controls the overall operation of the computer system.
\paragraph{Platform}

\paragraph{ROOT}
A framework for data processing, particularly for particle physicists. ROOT was developed by CERN C++.


\paragraph{RPC}
Remote procedure call. Execute a routine in another address space without coding details for the remote interaction
\paragraph{Stream}

\paragraph{Web server}

\paragraph{Windows}
A DOS based operating system created by Microsoft.

\pagebreak[4]

<<<<<<< HEAD
\section{Links}
The Github repository for this document can be found at \url{https://github.com/rootjs/specifications}. \\
The repository for the project itself is located at \url{https://github.com/rootjs/rootjs}.
=======
The Github repository for this document can be found at  \href{here}{https://github.com/rootjs/specifications}, the repository for the project itself at \href{here}{https://github.com/rootjs/rootjs}.
>>>>>>> e153b9f244038d72d36219bd7f53a89b2344d1f9
