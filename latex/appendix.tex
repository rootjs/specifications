\chapter{Appendix}
\section{Glossary}
\paragraph{Application programming interface (API)}
A collection of tools, libraries and documentation for creating software. APIs expose functionality through specified interfaces, which allows to develop independent from the specific implementation

\paragraph{Asynchronous I/O}
Using asynchronous I/O, an application does not have to wait for the data transmission to finish, but can continue in execution. This prevents leaving the CPU idle, as I/O operations are usually significantly slower than other tasks.

\paragraph{C++}
A general purpose, object oriented programming language and one of the most widely used programming languages.

\paragraph{Compiler}
A computer program that translates source code into another language.

\paragraph{Cling}
An interactive C++ interpreter with a command line prompt that offers just-in-time compilation.

\paragraph{CERN}
The European Organization for Nuclear Research.

\paragraph{enum}
A data structure (enumeration) that contains named elements that can be compared to each other. Each element maps to a constant value.

\paragraph{Exception}
Exception handling is the process of responding to the occurrence, during computation, of exceptions – anomalous or exceptional conditions requiring special processing - often changing the normal flow of program execution.\footnote{\url{https://en.wikipedia.org/wiki/Exception_handling}}

\paragraph{Framework}
Extendable, generic software to be extended by user-written modules.

\paragraph{Git}
A Version Control System (VCS).

\paragraph{Github}
A host for Git repositories.

\paragraph{Google V8 JavaScript}
An open-source JavaScript interpreter.

\paragraph{HTTP}
The HyperText Transfer Protocol, used for data communication on the Internet.

\paragraph{Interpreter}
A computer program that directly executes written instructions, without compiling them first.

\paragraph{JavaScript}
An interpreted programming or script language created by Netscape.

\paragraph{JavaScript Object Notation (JSON)}
A data format used for exchanging (data) objects between applications. The JSON format is easy to read and write for humans.

\paragraph{Just-in-time (JIT) compilation}
Compilation method that tries to speed up interpreter based program execution by dynamically compiling the executed program (parts) at run time.

\paragraph{Language bindings}
Expose an API to another programming language. This enables usage of low-level functionality in higher-level environments (such as the ROOT API in Node.js using the rootJS bindings).

\paragraph{Large Hadron Collider (LHC)}
World's largest particle collider located in CERN.

\paragraph{Linux}
A free and open source computer operating system.

\paragraph{Low Level Virtual Machine (LLVM)}
A compiler infrastructure written in C++.

\paragraph{Mac}
A Unix based operating system created by Apple.

\paragraph{Method overloading}
Declaring a method multiple times within the same namespace using different parameters.

\paragraph{Method signature}
A unique declaration which defines the name and the parameter list of a method.

\paragraph{Node.js}
A runtime environment for developing server-side web applications written in JavaScript.

\paragraph{Object}
An instance of a class containing data, such as variables, functions and data structures.

\paragraph{Operating System}
A piece of software managing software and hardware resources, input/output, and also controls the overall operation of the computer system.

\paragraph{Web Graphics Library (WebGL)}
WebGL enables hardware accelerated rendering of 3D content for web browsers.

\paragraph{ROOT}
A framework for data processing; developed at CERN and particularly used by particle physicists.

\paragraph{Remote procedure call (RPC)}
Execute a routine in another address space without coding details for the remote interaction.

\paragraph{Stream}
Potentially unlimited sequence of data elements made available over time.\footnote{\url{https://en.wikipedia.org/wiki/Stream_(computing)}}

\paragraph{Version Control System (VCS)}
Version Control Systems are used to track and (if necessary) revert changes made on a software project.

\paragraph{Windows}
A DOS based operating system created by Microsoft.

\pagebreak[4]

\section{Links}

The Github repository for this document may be found at \url{https://github.com/rootjs/specifications}. \\
The repository for the rootJS project itself is located at \url{https://github.com/rootjs/rootjs}.
