\chapter{Appendix}
\section{Glossary}
\paragraph{Application programming interface}
\paragraph{Asynchronous}
\paragraph{Binding API}
\paragraph{C++}
It is a general purpose, object oriented programming language and is one of the most widely used programming languages.
\paragraph{Cling}
An interactive C++ interpreter with a command line prompt and uses a just-in-time compiler.
\paragraph{enums}
\paragraph{European Organization for Nuclear Research (CERN)}
\paragraph{Exception}
\paragraph{Expose}
\paragraph{Framework}
\paragraph{Google V8 JavaScript}
\paragraph{HTTP}
\paragraph{Input/Output (I/O)}
\paragraph{Interpreter}
\paragraph{JavaScript}
An interpreted programming or script language created by Netscape.
\paragraph{JavaScript Object Notation (JSON)}
A data format which is easy to read and write for humans.
\paragraph{Just-in-time (JIT) compilation}
\paragraph{Large Hadron Collider (LHC)}
\paragraph{Linux}
A free and open source computer operating system.
\paragraph{Low Level Virtual Machine (LLVM)}
A compiler infrastructure written in C++.
\paragraph{Mac}
A Unix based operating system created by Apple.
\paragraph{Method overloading}

\paragraph{Method signature}

\paragraph{node.js}
A runtime environment for developing server-side web applications written in Javascript. 

\paragraph{Object}

\paragraph{Operating System}
A piece of software managing software and hardware resources, input/output, and also controls the overall operation of the computer system.
\paragraph{Platform}

\paragraph{ROOT}
A framework for data processing, particularly for particle physicists. ROOT was developed by CERN C++.


\paragraph{RPC}

\paragraph{Stream}

\paragraph{Web server}

\paragraph{Windows}
A DOS based operating system creating by Microsoft.