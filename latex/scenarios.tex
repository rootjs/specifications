\section{Scenarios}

Our bindings do not add to the core functionality of ROOT. Therefore 
we decided to give some examples on how the bindings might be used.

\begin{figure}[htb]
	\centering
	\begin{longtable}{p{3cm} @{\hskip 1cm} p{12cm}}
		\hline
		
		\textit{Scenario Name} & \underline{WebViewer}\\
		\hline
	
		\textit{Abstract} & A browser based GUI for realtime representation of root graphs.\\
		\hline
	
		\textit{Participating actor instances} & \underline{WebViewer:Node.js}; \underline{:ROOT}; \underline{:rootJS}\\
		\hline
	
		\textit{Flow of events} & 
		\begin{enumerate}
			\item rootJS is up and running initialize has already been executed.
			
			\item WebViewer calls the API method to get graphical output of the data ROOT has currently loaded.
				\begin{enumerate}
					\item rootJS processes the request and calls the corresponding ROOT functionality.
					\item rootJS receives ROOT output and streams it to WebViewer
				\end{enumerate}
			\item WebViewer uses the provided data to display the graph on its GUI
				\begin{enumerate}
					\item Node.js invokes ROOT I/O operations.
						\begin{enumerate}
							\item ROOT loads data and provides raw visualization data.
						\end{enumerate}
					\item Node.js serializes data and streams it to WebViewer.
				\end{enumerate}
			\item WebViewer receives data and renders it in the browser.
		\end{enumerate}
		\\
		\hline
		
	\end{longtable}
	
	\caption{WebViewer scenario}
	
\end{figure}

\begin{figure}[htb]
	\centering
	\begin{longtable}{p{3cm} @{\hskip 1cm} p{12cm}}
		\hline
		\textit{Scenario Name} & \underline{eventViewer}\\
		\hline
		\textit{Abstract} &
		EventViewer providing a visualisation of experimental data, showing signals particles have produced in the detector.
		The visualised data can come both from acquired files (which in turn can contain either real data at various degrees of processing or the output of Monte-Carlo simulations) and directly from the detector.
		This kind of visualisation is typically fairly qualitative but it is very useful to perform a quick eyescan of data, or - quite a common scenario - to monitor live whether the detector actually acquires data properly.
		\\
		\hline
		\textit{Participating actor instances} & 
		\underline{:}\\
		\hline
		\textit{Flow of events} &
		\begin{enumerate}
			\item data acquisition
			\item preprocessing
			\item 3D visualisation
			\item stream live data from the detector
			\begin{enumerate}
				\item access the data source (which can use dedicated readout hardware, and even if it is on the network it is typically highly access-restricted)
				\item process incoming data in a timely manner
			\end{enumerate}
			\item publish its data e.g. as JSON over RESTful HTTP or WebSockets
			\item web browser for visualisation itself and interaction with the user.
			\item render it locally e.g. using WebGL.
		\end{enumerate}
		\\
		\hline
	\end{longtable}
	\caption{The X scenario for the use case Y}
\end{figure}
