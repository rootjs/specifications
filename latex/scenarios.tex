\section{Scenarios}

Our bindings do not add to the core functionality of ROOT. Therefore 
we decided to give some examples on how the bindings might be used.

\begin{figure}[htb]
	\centering
	\begin{longtable}{p{3cm} @{\hskip 1cm} p{12cm}}
		\hline
		
		\textit{Scenario Name} & \underline{WebViewer}\\
		\hline
	
		\textit{Abstract} & A browser based GUI for realtime representation of root graphs.\\
		\hline
	
		\textit{Participating actor instances} & \underline{WebViewer:Node.js}; \underline{:ROOT}; \underline{:rootJS}\\
		\hline
	
		\textit{Flow of events} & 
		\begin{enumerate}
			\item rootJS is up and running initialize has already been executed.
			
			\item WebViewer calls the API method to get graphical output of the data ROOT has currently loaded.
				\begin{enumerate}
					\item rootJS processes the request and calls the corresponding ROOT functionality.
					\item rootJS receives ROOT output and streams it to WebViewer
				\end{enumerate}
			\item WebViewer uses the provided data to display the graph on its GUI
				\begin{enumerate}
					\item Node.js invokes ROOT I/O operations.
						\begin{enumerate}
							\item ROOT loads data and provides raw visualization data.
						\end{enumerate}
					\item Node.js serializes data and streams it to WebViewer.
				\end{enumerate}
			\item WebViewer receives data and renders it in the browser.
		\end{enumerate}
		\\
		\hline
		
	\end{longtable}
	
	\caption{WebViewer scenario}
	
\end{figure}

\begin{figure}[htb]
	\centering
	\begin{longtable}{p{3cm} @{\hskip 1cm} p{12cm}}
		\hline
		\textit{Scenario Name} & \underline{EventViewer}\\
		\hline
		\textit{Abstract} &
		A Web based event viewer providing a visualisation of experimental data, showing signals particles have produced in the detector.
		The web viewer is split into the back-end, server part, with access to
	    the data source and enough resources to process the data, and the front-end, client part, that is a modern-enough Web browser, and responsible only for visualisation itself and interaction with the user.
		\\
		\hline
		\textit{Participating actor instances} & 
		\underline{Server:Node.js Application}; \underline{:ROOT}; \underline{EventViewer:Browser}; \underline{:rootJS}\\
		\hline
		\textit{Flow of events} &
		\begin{enumerate}
			\item EventViewer requests visual updates from Server.
			\item Server interfaces with ROOT via rootJS.
			\item ROOT acquires data from external source.
			\begin{enumerate}
					\item External, dedicated readout hardware is used to access the data source.
					\item ROOT processes incoming data in a timely manner.
			\end{enumerate}
			\item ROOT passes the data prepared for (3D) visualisation to the Server via rootJS.
			\item Server publishes its data as JSON stream over the web.
			\item EventViewer renders received data locally e.g. using WebGL.
		\end{enumerate}
		\\
		\hline
	\end{longtable}
	\caption{EventViewer scenario}
\end{figure}
