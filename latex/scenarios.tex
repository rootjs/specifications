\section{Scenarios}

\begin{figure}[htb]
	\centering
	\begin{longtable}{p{3cm} @{\hskip 1cm} p{12cm}}
		\hline
		\textit{Scenario Name} & test\\
		\hline
		\textit{Participating actor instances} & test\\
		\hline
		\textit{Flow of events} & 
			\begin{enumerate}
				\item event
				\item action
				\item event
			\end{enumerate}
			\\
		\hline
	\end{longtable}
	\caption{The X scenario for the use case Y}
\end{figure}


%The scenarios are based on Node.js based web applications on a server running ROOT and rootJS\\
%Bindings do not extend the basic functionality of the underlying framework.\\
%We therefore only assume a fictional application, using our API, to be the user of rootJS.\\
%
%
%Scenario 1: //TODO fix layout and create UML\\
%WebViewer, a browser based GUI for realtime representation of root graphs
%	rootJS is up and running initialize has already been executed
%	WebViewer calls the API method to get graphical output of the data ROOT has currently loaded
%	\indent rootJS processes the request and calls the corresponding ROOT functionality
%	\indent rootJS receives ROOT output and streams it to the Webviewer client
%webViewer uses the provided data to display the graph on its GUI
%
%\indent	Node.js invokes ROOT I/O operations\\
%\indent \indent		ROOT loads data and provides raw visualization data\\
%\indent	Node.js serializes data and streams it to the web viewer\\
%Web viewer receives data and renders it in the browser\\
%