\section{Scenarios}

The bindings do not affect core functionality of ROOT. Therefore 
we decided to give some examples on how the bindings might be used. The 
user in this case is a Node.js based application, that uses the rootJS bindings 
to interact with ROOT. As the actual implementation is not yet 
finalized, the scenarios only contain samples of simple interactions 
between an Application and ROOT.

\subsection{Web Viewer Scenario}
\begin{figure}[htb]
	\centering
	\begin{longtable}{p{3cm} @{\hskip 1cm} p{12cm}}
		\hline
		
		\textit{Scenario name} &  \texttt{WebViewer}\\
		\hline
	
		\textit{Abstract} & A browser based GUI for realtime representation of \texttt{ROOT} graphs.\\
		\hline
	
		\textit{Participating actor instances} & \underline{WebViewer:Node.js}; \underline{:ROOT}; \underline{:rootJS}\\
		\hline
	
		\textit{Flow of events} & 
		\begin{enumerate}
			\item \texttt{rootJS} is up and running initialize has already been executed.
			
			\item \texttt{WebViewer} calls the API method to get graphical output of the data \texttt{ROOT} has currently loaded.
				\begin{enumerate}
					\item \texttt{rootJS} processes the request and calls the corresponding \texttt{ROOT} functionality.
					\item \texttt{rootJS} receives \texttt{ROOT} output and streams it to \texttt{WebViewer}.
				\end{enumerate}
			\item \texttt{WebViewer} uses the provided data to display the graph on its GUI.
				\begin{enumerate}
					\item \texttt{Node.js} invokes \texttt{ROOT} I/O operations.
						\begin{enumerate}
							\item \texttt{ROOT} loads data and provides raw visualization data.
						\end{enumerate}
					\item \texttt{Node.js} serializes data and streams it to \texttt{WebViewer}.
				\end{enumerate}
			\item \texttt{WebViewer} receives data and renders it in the browser.
		\end{enumerate}
		\\
		\hline
		
	\end{longtable}
	
	\caption{WebViewer scenario}
	
\end{figure}

\pagebreak[4]

\subsection{Event Viewer Scenario}
\begin{figure}[htb]
	\centering
	\begin{longtable}{p{3cm} @{\hskip 1cm} p{12cm}}
		\hline
		\textit{Scenario name} & \texttt{EventViewer}\\
		\hline
		\textit{Abstract} &
		A Web based event viewer providing a visualisation of experimental data, showing signals particles have produced in the detector.
		The web viewer is split into the back-end, server part, with access to
	    the data source and enough resources to process the data, and the front-end, client part, that is a modern-enough Web browser, and responsible only for visualisation itself and interaction with the user.
		\\
		\hline
		\textit{Participating actor instances} & 
		\underline{Server:Node.js Application}; \underline{:ROOT}; \underline{EventViewer:Browser}; \underline{:rootJS}\\
		\hline
		\textit{Flow of events} &
		\begin{enumerate}
			\item \texttt{EventViewer} requests visual updates from \texttt{Server}.
			\item \texttt{Server} interfaces with \texttt{ROOT} via \texttt{rootJS}.
			\item \texttt{ROOT} acquires data from external source.
			\begin{enumerate}
					\item External, dedicated readout hardware is used to access the data source.
					\item \texttt{ROOT} processes incoming data in a timely manner.
			\end{enumerate}
			\item \texttt{ROOT} passes the data prepared for (3D) visualisation to the \texttt{Server} via \texttt{rootJS}.
			\item \texttt{Server} publishes its data as JSON stream over the web.
			\item \texttt{EventViewer} renders received data locally e.g. using WebGL.
		\end{enumerate}
		\\
		\hline
	\end{longtable}
	\caption{EventViewer scenario}
\end{figure}