\section{Usage}
\begin{frame}{Product usage}
        rootJS will be used to create web-applications that can:
        \begin{itemize}
        	\item Expose processed data (that might otherwise be hard to access) and then visualize it locally
        	\item Interact with data both stored somewhere accessible for the server or streamed via remote procedure call (RPC)
        	\item Run on any platform that supports a browser
        \end{itemize}
\end{frame}


\subsection{Audience}
\begin{frame}{Audience}
        Most users of rootJS will be used to working in Linux and with web servers. At the very least, they will be able to install ROOT
        and also be proficient in programming languages like JavaScript and C++.
        \begin{itemize}
        	\item Scientists (e.g. particle physicists)
        	\item Researchers
        	\item Web-developers interested in creating applications based on ROOT
        \end{itemize}
\end{frame}

\subsection{Operating conditions}
\begin{frame}{Operating conditions}
        \begin{itemize}
                \item rootJS will be used on servers that run ROOT and have access to the required data sources.
                \item As ROOT 6 currently runs on Linux and OS X only, usage of the bindings is limited to those platforms.
        \end{itemize}
\end{frame}
